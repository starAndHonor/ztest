\documentclass{article}
\usepackage{ctex} % 加入ctex宏包以支持中文
\usepackage{graphicx} % 用于插入图片
\usepackage{geometry} % 用于设置页面布局
\usepackage{booktabs} % 用于美化表格
\usepackage{enumitem} % 用于自定义列表格式
\usepackage{lipsum}  % 导入生成段落的宏包
\usepackage{ctex} % 加入ctex宏包以支持中文
\usepackage{graphicx} % 加载图形处理宏包
\usepackage{booktabs} % 用于美化表格线条
\usepackage{caption} % 用于添加表格标题
\usepackage[table]{xcolor} % 用于设置表格颜色
\usepackage{float}

\usepackage{subcaption}
\usepackage{tabularx}
\usepackage{tcolorbox}
\tcbuselibrary{listings}
\usepackage{array}
\usepackage[dvipsnames]{xcolor}
\usepackage{tikz}
\usepackage{hyperref} 
\usepackage{framed}
\usetikzlibrary{positioning} % 加载positioning库
\renewcommand{\refname}{}

\hypersetup{
    colorlinks=true, % 使用颜色而不是边框
    linkcolor=black, % 链接颜色为黑色
    citecolor=black, % 引用颜色为黑色
    urlcolor=blue    % URL颜色为蓝色(可以根据需要修改)
}
\lstset{
 basicstyle=\ttfamily, % 设置字体族
 breaklines=true, % 自动换行
 keywordstyle=\bfseries\color{NavyBlue}, % 设置关键字为粗体,颜色为 NavyBlue
 morekeywords={}, % 设置更多的关键字,用逗号分隔
 emph={self}, % 指定强调词,如果有多个,用逗号隔开
    emphstyle=\bfseries\color{Rhodamine}, % 强调词样式设置
    commentstyle=\itshape\color{black!50!white}, % 设置注释样式,斜体,浅灰色
    stringstyle=\bfseries\color{PineGreen!90!black}, % 设置字符串样式
    columns=flexible,
    % numbers=left, % 显示行号在左边
    % numbersep=2em, % 设置行号的具体位置
    % numberstyle=\footnotesize, % 缩小行号
    % frame=single, % 边框
    framesep=1em, % 设置代码与边框的距离
	tabsize=2, % 设置tab键的宽度为2个空格
    showstringspaces=false, % 不显示字符串中的空格
}

\geometry{a4paper, margin=1in}
\begin{document}

\begin{titlepage}
    \centering
    \vspace*{1cm}

    % 插入学校标志
    \includegraphics[width=\textwidth]{img/logo.png} % 假设logo.png是学校标志的图片文件

    % \vspace{1.5cm}
    \fontsize{24pt}{32pt}\selectfont
    \textbf{《High-level Language Programming Project》 Report}
    \vspace{4cm}

    \centering
    Ztest: A C++ Unit Testing Framework
    \vspace{2cm}

    \fontsize{16pt}{16pt}\selectfont
    \begin{tabular}{ll}                                                                \\
        School       :   & School of Future Technology\hspace{6cm}          \\
        Major        :   & Data Science and Big Data Technology\hspace{6cm} \\
        Student Name :   & Zheng Chenyang,Ye Suohua,Wu Hongqing,            \\
                         & Qi Yansong,Wang Ruizhen                          \\
        Teacher      :   & Zhang Huaidong\hspace{6cm}                       \\
        Submission Date: & 2025.6.6\hspace{6cm}                             \\
    \end{tabular}

    \vfill

    \vspace{1cm}
\end{titlepage}

\tableofcontents  % 生成目录表
\newpage
\section{系统需求分析}
\subsection{系统背景与动机}
% 在这里填写系统背景和动机的内容
随着复杂系统架构设计能力的培养需求日益凸显,掌握面向对象设计原则与模式化工程实践已成为高级软件工程教育的核心目标。这不仅需要理解类与对象间的动态协作关系,更要具备通过设计模式解决架构难题的抽象思维能力。
现有的单元测试框架(如Google Test)具有上手难度较高,并发支持有限,报告系统过于简洁,可拓展性一般等缺点,我们团队计划开发一个提供一个灵活、高效且易于使用(带图形用户界面GUI)的测试工具,旨在提供一个直观、易用的环境,方便开发人员和测试人员编写、运行和管理测试用例。该工具将支持多种测试类型(如单元测试、集成测试等),并提供详细的测试结果报告。
\begin{table}[h]
    \centering
    \caption{主流测试框架对比}
    \label{tab:compare}
    \begin{tabularx}{\textwidth}{lXXXXX}
        \toprule
        \textbf{框架}              & \textbf{GUI支持} & \textbf{并发测试} & \textbf{报告系统} & \textbf{扩展性} & \textbf{数据驱动} \\
        \midrule
        Google Test(C++)         & 无              & 有限            & 基础            & 中等           & 不支持           \\
        JUnit    (Java)          & Eclipse插件      & 支持            & HTML/XML      & 高            & 支持            \\
        PyTest (Python)          & 第三方工具          & 优秀            & 丰富            & 优秀           & 支持            \\
        Catch2   (C++)           & 无              & 一般            & 简洁            & 中等           & 不支持           \\
        % \midrule
        \textbf{Ztest*}    (C++) & 精美             & 优秀            & 丰富            & 高            & 支持            \\
        \bottomrule
    \end{tabularx}
\end{table}
\subsection{系统目标}
\begin{figure}[H]
    \centering
    \includegraphics[width=0.6\textwidth]{img/func.png} % 假设logo.png是学校标志的图片文件
    \caption{ ztest function}
    \label{fig:ztest function }
\end{figure}
\subsubsection{测试管理}
% 我们将任务分为四个类别:短测试时间且更关注结果正确性的任务,如加法运算、字符串拼接和用户登录验证;长测试时间且更关注结果正确性的任务,如合并多个文件、复杂字符串匹配与替换和大量数据排序结果验证;短测试时间且更关注运行过程评估的任务,如读取或写入大文件、低复杂度算法性能测试和数据库查询;以及长测试时间且更关注运行过程评估的任务,如多线程处理任务、压力测试和时间复杂度高算法测试。

% \begin{figure}[H]
%     \centering
%     \includegraphics[width=0.6\textwidth]{img/task.png} % 假设logo.png是学校标志的图片文件
%     \caption{ task type}
%     \label{fig:ztest task type }
% \end{figure}

相较传统的单元测试框架,Ztest支持多种测试类型,如可并行运行且线程安全测试、需串行运行或线程不安全测试、通过迭代评估性能、参数化数据驱动测试。我们提供测试夹具用来管理单个测试。

\begin{figure}[H]
    \centering
    \includegraphics[width=0.8\textwidth]{img/types.png} % 假设logo.png是学校标志的图片文件
    \caption{ test type}
    \label{fig:test types }
\end{figure}
我们提供类似Google Test的宏简化测试定义:
\begin{framed}
    \begin{lstlisting}[language=C++]
ZTEST_F(SuiteName, TestName, safe/unsafe) { ... } //define a test 
ZBENCHMARK(SuiteName, TestName, iterations) { ... } // define a benchmark
ZTEST_P(SuiteName, TestName, data) { ... }  //define a parameterized test
ZTEST_P_CSV(SuiteName, TestName, "data.csv") { ... } // define a parameterized test with csv data
\end{lstlisting}
\end{framed}

同时,我们能够链式定义测试用例:
\begin{framed}
    \begin{lstlisting}[language=C++]
auto test_case = TestFactory::createTest("Add", ZType::Z_SAFE, "", add, 2, 3)
                 .setExpectedOutput(5)
                 .beforeAll([]() { logger.info("Init\n"); })          
                 .afterEach([]() { logger.info("Clean\n"); }).build();
                \end{lstlisting}
\end{framed}

\subsubsection{断言机制}
断言机制用于验证测试用例的预期结果是否正确。框架提供了多种断言宏,如 \textbf{EXPECT\_EQ} 和 \textbf{ASSERT\_TRUE},支持在测试中快速验证条件。
主要提供以下断言:
\begin{itemize}
    \item \textbf{EXPECT\_EQ}:验证两个值是否相等。
    \item \textbf{ASSERT\_TRUE}:验证条件是否为真。
\end{itemize}
使用样例如下,
\begin{lstlisting}[language=C++]
// 如果断言失败,抛出异常
EXPECT_EQ(5, add(2, 3));
ASSERT_TRUE(6==add(2, 3));
\end{lstlisting}

如果断言失败,抛出 \textbf{ZTestFailureException} 异常。同时可以通过继承 \textbf{ZTestFailureException} 异常处理函数,自定义异常处理逻辑。
\subsubsection{数据驱动测试}
参数化测试是指在测试过程中,将测试用例中某些固定的数据替换为参数,然后通过改变参数的值来生成多组测试数据,从而对软件进行多次测试的方法。它允许测试人员用一组测试逻辑覆盖多种输入情况,而不是为每种情况编写单独的测试代码。
数据驱动测试框架(Data-Driven Testing Framework)是一种将测试数据与测试脚本分离的自动化测试框架。它通过外部数据源(如Excel、CSV文件、数据库等)来驱动测试用例的执行。

ZTest实现了参数测试并以从csv导入和解析数据的方式实现了数据驱动的测试。为了加速数据的访问,缓存导入的数据。使用懒加载机制,只有在需要时才加载数据到缓存中,减少不必要的资源消耗。同时,使用了LRU缓存淘汰策略用来确保缓存中的数据是最有可能被再次访问的,从而最大限度地发挥缓存的作用。

\subsubsection{测试执行器}
测试执行器负责管理测试用例的运行,支持\textbf{多线程}并行执行测试用例,并收集测试结果,实现了自动调度测试和测试结果统计。
他的行为如下:
\begin{itemize}
    \item 并行运行safe测试。线程的创建和销毁是一个相对耗时的操作。使用线程池管理线程的生命周期通过复用线程,避免了频繁的线程创建和销毁,从而显著提高了系统的性能。
    \item 串行运行unsafe测试。通过队列来维护测试用例,保证测试用例的顺序执行。
    \item 串行运行Benchmark测试。
    \item 串行运行Parameterized测试。
\end{itemize}
\subsubsection{报告生成}
能够生成HTML,JSON和XML报告,方便查看测试结果和用于CI/CD。
\begin{figure}[H]
    \centering
    \includegraphics[width=0.8\textwidth]{img/report.png}
    \caption{测试结果展示界面布局示意图}
    \label{fig:report}
    \small
\end{figure}
\subsubsection{CLI接口}
\begin{framed}
    \begin{lstlisting}
Usage: executor_name [OPTIONS] 
Options: 
--help                         Show help 
--run-all                      Run all tests
--run-safe                     Run safe tests
--run-unsafe                   Run unsafe tests
--run-benchmark                Run benchmark tests
--run-parameterized            Run parameterized tests
--run-test-case TEST_CASE      Run a specific test case
\end{lstlisting}
\end{framed}

\subsubsection{GUI展示}
可以通过GUI展示测试结果,筛选测试结果,观看系统资源状态,查看某个测试的运行细节等功能。
\begin{figure}[H]
    \centering
    \includegraphics[width=\textwidth]{img/gui.png}
    \caption{测试管理界面布局示意图}
    \label{fig:gui}
    \small
\end{figure}
\subsubsection{AI智能诊断功能}
通过调用qwen3的接口,实现智能诊断功能,在报告中可以得到
\begin{enumerate}
    \item 识别失败根本原因
    \item 提供修复建议
    \item 指出高风险测试用例
    \item 评估整体测试覆盖率
    \item 提出系统稳定性改进建议
\end{enumerate}

同时可以在GUI界面上获取对于某一个测试用例的具体分析,可以自定义提示词和上传源文件帮助分析。
\newpage
\section{程序分析}
\subsection{ 系统关键问题}
\subsubsection{如何进行优秀的架构设计?}
ztest使用了核心模块分层架构,具体如下
\begin{itemize}
    \item 基础层:提供测试框架的基本功能,如测试用例注册、测试执行、测试结果管理、日志输出等。
    \item 执行层:负责测试用例的运行,包括测试用例的注册、执行、结果管理、日志输出等。
    \item 数据层:实现参数化测试数据驱动,提供测试用例的数据源。
    \item 报告层:提供多格式报告输出,如HTML、JSON、XML等。
    \item 并发层:封装线程池管理并行执行,提高测试效率。
\end{itemize}

在实现过程中ztest使用了面向对象编程,泛型编程,函数式编程,并发编程,元编程等五种编程范式。

\subsubsection{如何提升提升测试运行效率?}
传统的C++测试框架(如gtest,catch2)多使用顺序运行测试的方法执行所有测试,导致测试时间较长,降低了生产效率。同时vibe coding的出现导致编写测试用例的时间大大降低,而运行测试用例的时间仍然没有大的降低。
我们将任务分为四个类别:短测试时间且更关注结果正确性的任务,如加法运算、字符串拼接和用户登录验证;长测试时间且更关注结果正确性的任务,如合并多个文件、复杂字符串匹配与替换和大量数据排序结果验证;短测试时间且更关注运行过程评估的任务,如读取或写入大文件、低复杂度算法性能测试和数据库查询;以及长测试时间且更关注运行过程评估的任务,如多线程处理任务、压力测试和时间复杂度高算法测试。
\begin{figure}[H]
    \centering
    \includegraphics[width=0.8\textwidth]{img/task.png} % 假设logo.png是学校标志的图片文件
    \caption{ task type}
    \label{fig:task types }
\end{figure}
对于测试所需时间长,更见关注结果正确性,且线程安全的任务,ZTest在C++测试框架体系中引入了多线程测试执行,直接融入测试框架,而不是google-parallel这样的第三方工具,实现了更高细粒度的控制。
同时,引入数据驱动测试并使用缓存加速使得大规模的单元测试的定义和执行速度大大提升,意义重大。
\subsubsection{如何实现语法糖?}
如何让用户愿意使用单元测试框架?其关键在于语法定义的简易性。我们使用一系列复杂的宏定义来实现语法糖,从而简化用户的定义。在实现链式语法的时候,我们每个函数调用返回一个对象,将多个函数调用或操作连接在一起。
\subsubsection{如何实现待测函数的自动类型推导?}
我们设计在使用单个测试样例链式定义的时候,我们希望用户只需要传入函数名、参数和期望结果,就可以由为我们的框架接管运行的具体逻辑和测试结果的比较。
这要求我们实现自动推导待测函数的参数类型,以便调用函数用于测试和测试结果验证。我们使用工厂模式来实现返回值类型的指定,使用建造者模式来实现参数的构造和期望结果的设定。
\subsection{ 职责分配}
% 在这里填写每个学生的职责分配
\begin{itemize}[leftmargin=*]
    \item 郑辰阳: 架构设计,ztest core和gui主要代码编写,报告撰写和答辩
    \item 叶穗华: GUI的改进,部分测试逻辑的改进
    \item 吴泓庆:GUI的改进
    \item 齐彦松: GUI的改进
    \item 王瑞箐:迁移到windows上的尝试
\end{itemize}

\section{技术路线}
\subsection{运行环境}
\begin{table}[H]
    \centering
    \caption{开发和运行环境}
    \begin{tabular}{@{}>{\bfseries}c>{\raggedright\arraybackslash}p{10cm}@{}} % 设置第一列为加粗,第二列为左对齐
        \toprule
        \textbf{Component} & \textbf{Tool Used for Development}                    \\
        \midrule
        处理器                & Intel i9-14900HX (32) @ 5.800GHz                      \\
        操作系统               & Ubuntu 24.04.2 LTS x86\_64 (kernel 6.11.0-26-generic) \\
        编译器                & gcc 13.3.0 或 clang 18.1.3 (不可以使用MSVC)                 \\
        图形API              & glfw 3.4 + glad 4.0.1                                 \\
        GUI框架              & ImGui-1.91.7-docking                                  \\
        数据可视化工具            & implot v0.16                                          \\
        构建系统               & XMake v2.9.9+HEAD.40815a0                             \\
        C++标准              & C++20(必要)                                             \\
        \bottomrule
    \end{tabular}
\end{table}
\subsection{总体设计}
\subsubsection{Ztest Core架构图}
% 在这里描述系统的总体设计
\begin{figure}[H]
    \centering
    \includegraphics[width=0.8\textwidth]{img/core.png} % 假设logo.png是学校标志的图片文件
    \caption{ ztest design}
    \label{fig:ztest design }
\end{figure}
ZTest的核心架构流程开始于测试定义阶段,测试(包括ZTest Suite、安全测试、不安全测试、基准测试和参数化测试)通过会使用断言定义,如EXPECT\_EQ和ASSERT\_TRUE,来验证测试结果的正确性。
接着,测试将被注册到ZTestRegistry进行统一的管理。
注册完成后,测试会被发送到测试执行器,该执行器根据测试的类型采取不同的测试策略,对于安全测试采用并行测试策略,而对于其他类型的测试则采用串行测试策略。
在对数据驱动的测试的执行过程中,数据驱动模块通过ZDataRegistry(带有LRU机制的缓存)来管理外部数据,如CSV文件,以支持测试的进行。
测试完成后,结果管理器ZTestResultManager负责收集和处理测试结果,并将这些结果导出为HTML、JSON或XML格式,以便于报告和进一步分析。
在导出为HTML格式时由AI(qwen turbo)进行测试诊断,并融入到HTML测试报告中。
此外,整个测试流程还支持与持续集成/持续部署(CI/CD)系统的集成,使得测试结果可以自动地反馈到开发流程中,从而提高软件开发的效率和质量。
\begin{figure}[H]
    \centering
    \includegraphics[width=0.8\textwidth]{img/guiarch.png} % 假设logo.png是学校标志的图片文件
    \caption{ ztest gui architecture}
    \label{fig:ztest gui architecture }
\end{figure}
\subsubsection{GUI框架}
GUI框架使用MCV架构开发,其中Model层负责数据处理,其中包含了对于Ztest core的进一步封装,View层负责界面绘制,主要是使用ImGui框架绘制,Controller层负责用户交互,将用户在UI界面上的点击和筛选等操作转换成对Model层和View层的调用,从而实现用户界面的更新和数据的更新。
\subsubsection{整体类图}
\begin{figure}[H]
    \centering
    \includegraphics[angle=270,width=0.6\textwidth]{img/class.png} % 假设logo.png是学校标志的图片文件
    \caption{ ztest class}
    \label{fig:ztest class }
\end{figure}
\subsection{详细设计}
\subsubsection{测试定义相关类}
\begin{itemize}
    \item \texttt{ZTestInterface}:  测试接口,定义测试执行框架。
    \item \texttt{ZTestBase} (ztest\_base.hpp): 抽象测试基类,封装通用属性与生命周期钩子。采用模板方法模式定义测试执行框架,通过虚函数表实现策略模式,利用钩子函数实现装饰器模式扩展。
    \item \texttt{ZTestSingleCase} (ztest\_singlecase.hpp): 单例测试实现,使用模板方法模式执行测试逻辑,配合TestFactory体现工厂模式,支持链式配置符合建造者模式特征。
    \item \texttt{ZTestSuite}(ztest\_suite.hpp): 测试套件类,继承自ZTestBase,用于组织测试用例,实现模板方法模式,支持钩子函数扩展。
    \item \texttt{ZBenchMark} (ztest\_benchmark.hpp): 基准测试类,继承ZTestBase并重写run()方法,通过迭代执行测试函数实现基准测试。
    \item \texttt{ZTestParameterized} (ztest\_parameterized.hpp): 参数化测试基类,采用组合模式封装测试数据,通过run()方法实现数据驱动测试框架。
    \item \texttt{宏定义系统} (ztest\_macros.hpp): 通过宏定义实现语法糖,采用工厂模式自动生成测试类,利用命名连接技术实现自动化注册。
\end{itemize}
\begin{figure}[H]
    \centering
    \includegraphics[width = \textwidth]{img/c1.png} % 假设logo.png是学校标志的图片文件
    \caption{ ztest class}
    \label{fig:ztest class }
\end{figure}

\subsubsection{测试注册相关类}
\begin{itemize}
    \item \texttt{ZTestRegistry} (ztest\_registry.hpp): 单例模式实现的全局注册中心,管理测试用例集合并保障线程安全的注册/获取操作。
\end{itemize}
\begin{figure}[H]
    \centering
    \includegraphics[width = 0.7\textwidth]{img/c2.png} % 假设logo.png是学校标志的图片文件
    \caption{ ztest class}
    \label{fig:ztest class }
\end{figure}
\subsubsection{测试执行相关类}
\begin{itemize}
    \item \texttt{ZTestContext} (ztest\_context.hpp): 测试执行上下文,采用策略模式根据测试类型选择执行策略,通过线程池模式实现并行测试执行。
    \item \texttt{ZThreadPool} (ztest\_thread.hpp): 线程池实现对象池模式,使用生产者-消费者模式管理任务队列,通过future/promise机制实现异步执行监控。
\end{itemize}
\begin{figure}[H]
    \centering
    \includegraphics[width = 0.7\textwidth]{img/c3.png} % 假设logo.png是学校标志的图片文件
    \caption{ ztest class}
    \label{fig:ztest class }
\end{figure}
\subsubsection{数据管理相关类}
\begin{itemize}
    \item \texttt{ZDataRegistry} (ztest\_dataregistry.hpp): 数据缓存管理器,采用单例模式全局访问,通过LRU策略实现缓存淘汰。
    \item \texttt{ZDataManager}(ztest\_parameterized.hpp): 是数据管理类的抽象基类,提供数据管理接口,实现类似python中的迭代器的功能。
    \item \texttt{ZTestDataManager} (ztest\_parameterized.hpp)泛型类,实现了由用户在代码中指定的数据类型的测试数据管理功能。
    \item \texttt{ZTestCSVDataManager}(ztest\_parameterized.hpp) 双继承于\texttt{ZTestDataManager}和\texttt{ZDataManager},实现了从csv文件中读取测试数据功能并处理成便于参数化测试的形式。
\end{itemize}
\begin{figure}[H]
    \centering
    \includegraphics[width = \textwidth]{img/c4.png} % 假设logo.png是学校标志的图片文件
    \caption{ ztest class}
    \label{fig:ztest class }
\end{figure}
\subsubsection{结果管理相关类}
\begin{itemize}
    \item \texttt{ZTestResult} (ztest\_result.hpp): 值对象模式的测试结果类,封装测试状态、耗时等不可变数据。
    \item \texttt{ZTestResultManager} (ztest\_result.hpp): 单例模式实现的结果管理器,采用责任链模式处理结果存储与查询。
\end{itemize}
\begin{figure}[H]
    \centering
    \includegraphics[width = 0.7\textwidth]{img/c5.png} % 假设logo.png是学校标志的图片文件
    \caption{ ztest class}
    \label{fig:ztest class }
\end{figure}
\subsubsection{报告生成相关类}
\begin{itemize}
    \item \texttt{ZLogger} (ztest\_logger.hpp): 多格式报告生成器,应用模板方法模式定义报告生成流程,支持HTML/JSON/JUnit报告格式。
\end{itemize}
\begin{figure}[H]
    \centering
    \includegraphics[width =0.7\textwidth]{img/c6.png} % 假设logo.png是学校标志的图片文件
    \caption{ ztest class}
    \label{fig:ztest class }
\end{figure}
\subsubsection{工具模块相关类}
\begin{itemize}
    \item \texttt{ZTimer} (ztest\_timer.hpp): RAII模式实现的计时器,封装时间测量功能。
    \item \texttt{CSVStream}(ztest\_utils.hpp):实现了类似标准输入输出库的CSV流操作,支持读,写,打印基本信息。
\end{itemize}
\begin{figure}[H]
    \centering
    \includegraphics[width = \textwidth]{img/c7.png} % 假设logo.png是学校标志的图片文件
    \caption{ ztest class}
    \label{fig:ztest class }
\end{figure}
\subsubsection{GUI模块相关类}
\begin{itemize}
    \item \texttt{ZTestModel} (gui.hpp): MVC架构中的Model层,采用观察者模式监听测试状态变化。
    \item \texttt{ZTestController} (gui.hpp): MVC架构中的Controller层,实现命令模式封装测试执行操作。
    \item \texttt{ZTestView} (gui.hpp): MVC架构中的View层,使用桥接模式分离界面元素与实现。
\end{itemize}
\begin{figure}[H]
    \centering
    \includegraphics[width = \textwidth]{img/c8.png} % 假设logo.png是学校标志的图片文件
    \caption{ ztest class}
    \label{fig:ztest class }
\end{figure}
\section{编程进度}
\begin{table}[H]
    \centering
    \begin{tabular}{|l|l|l|l|}
        \toprule
        \textbf{任务阶段}   & \textbf{时间}           & \textbf{计划}                       & \textbf{实际情况} \\ \midrule
        \textbf{确定主题}   & 2024.04.26-2024.05.03 & 调查主题,提交提案。                        & 已完成           \\ \midrule
        \textbf{实现核心代码} & 2024.05.03-2024.05.05 & 实现了GUI和Safe test和Unsafe test的执行逻辑 & 已完成           \\ \midrule
        \textbf{重大性能优化} & 2024.05.05-2024.05.08 & 线程池优化                             & 已完成           \\ \midrule
        功能优化            & 2024.05.08-2024.05.11 & 完善JSON,HTML输出,JUnit格式输出,CI集成      & 已完成           \\ \midrule
        功能优化            & 2024.05.11-2024.05.13 & GUI引入了imgui Docking功能             & 已完成           \\ \midrule
        功能优化            & 2024.05.13-2024.05.15 & 增加CLI                             & 已完成           \\ \midrule
        \textbf{重大功能优化} & 2024.05.15-2024.05.24 & 增加BENCHMARK测试                     & 已完成           \\ \midrule
        功能优化            & 2024.05.24-2024.05.25 & 增加了设备状态监测和可视化                     & 已完成           \\ \midrule
        \textbf{重大功能优化} & 2024.05.25-2024.05.28 & 增加了参数化测试并增加了数据驱动                  & 已完成           \\ \midrule
        \textbf{重大性能优化} & 2024.05.28-2024.06.02 & 增加了对数据的缓存和LRU缓存淘汰技术               & 已完成           \\ \midrule
        \textbf{重大功能优化} & 2024.06.02-2024.06.04 & 添加了AI诊断                           & 已完成           \\ \midrule
        总结工作            & 2024.6.04-2024.06.05  & 跨平台移植 \& 报告撰写                     & 已完成           \\
        \bottomrule
    \end{tabular}
    \caption{编程进度}
\end{table}

\section{测试报告}
\subsection{功能测试\&系统测试}
\subsubsection{断言功能测试}
该测试旨在验证 EXPECT\_EQ、EXPECT\_NEAR 和 ASSERT\_TRUE 等断言宏在成功和失败情况下的行为。通过设计不同的测试用例,可以检查这些断言是否能够正确地识别预期结果与实际结果之间的匹配情况,从而确保测试框架的断言功能能够可靠地工作。
\begin{framed}
    \begin{lstlisting}[language=C++]
int add(int a, int b) { return a + b; }
double subtract(double a, double b) { return a - b; }
ZTEST_F(ASSERTION, FailedEXPECT_EQ) {
  EXPECT_EQ(6, add(2, 3));
  return ZState::z_success;
}
ZTEST_F(ASSERTION, SuccessEXPECT_EQ) {
  EXPECT_EQ(5, add(2, 3));
  return ZState::z_success;
}
ZTEST_F(ASSERTION, SuccessEXPECT_NEAR) {
  EXPECT_NEAR(2, subtract(5.0, 3.0), 0.001);
  return ZState::z_success;
}

ZTEST_F(ASSERTION, FailedEXPECT_NEAR) {
  EXPECT_NEAR(2, subtract(5.1, 3.0), 0.001);
  return ZState::z_success;
}

ZTEST_F(ASSERTION, FailedASSERT_TRUE) {
  ASSERT_TRUE(false);
  return ZState::z_success;
}
ZTEST_F(ASSERTION, SuccessASSERT_TRUE) {
  ASSERT_TRUE(true);
  return ZState::z_success;
}
\end{lstlisting}
\end{framed}
\begin{figure}[H]
    \centering
    \includegraphics[width=0.7\textwidth]{img/ass.png} % 假设logo.png是学校标志的图片文件
    \caption{ assertion function test}
    \label{fig:assertion function test}
\end{figure}
\subsubsection{测试管理测试}
本测试模块旨在验证测试管理系统的功能完整性与可靠性。通过定义多种测试用例,包括单个安全测试、单个不安全测试以及包含多个断言的测试套件,系统能够全面覆盖不同的测试场景。此外,利用动态测试用例构建功能,可以灵活地创建和注册新的测试用例,进一步增强了测试框架的可扩展性。测试过程中,通过内存分配、函数调用验证以及断言检查,确保了测试用例的正确执行与预期结果的一致性。同时,测试框架还提供了前置和后置钩子功能,用于在测试前后进行必要的设置和清理操作,保障了测试环境的稳定性和测试结果的准确性。
\begin{framed}
    \begin{lstlisting}[language=C++]
ZTEST_F(TESTMANAGE, safe_test_single_case, safe) {
  ASSERT_TRUE(true);
  return ZState::z_success;
}
ZTEST_F(TESTMANAGE, unsafe_test_single_case, unsafe) {
  ASSERT_TRUE(true);
  return ZState::z_success;
}
ZTEST_F(TESTMANAGE, test_suite) {
  const size_t MB100 = 100 * 1024 * 1024;
  auto ptr = std::make_unique<char[]>(MB100);
  ASSERT_TRUE(ptr != nullptr);
  EXPECT_EQ(3, subtract(5, 3));
  EXPECT_EQ(6, add(2, 3));
  return ZState::z_success;
}
void createSingleTestCase() {
  // Use TestBuilder to construct test
  auto test =
      TestFactory::createTest("AdditionTest",                // Test name
                              ZType::z_safe,                 // Execution
                              "Test addition functionality", // Description
                              add, 2, 3 // Function and arguments
                              )
          .setExpectedOutput(5) // Set expected result
          .beforeAll([]() {     // Setup hook
            std::cout << "Setting up single test..." << std::endl;
          })
          .afterEach([]() { // Teardown hook
            std::cout << "Cleaning up after test..." << std::endl;
          })
          .withDescription("Verify basic addition")
          .registerTest()
          .build(); // Register with test
}
// in main()
createSingleTestCase();

\end{lstlisting}
\end{framed}
\begin{figure}[H]
    \centering
    \includegraphics[width=0.7\textwidth]{img/manage.png} % 假设logo.png是学校标志的图片文件
    \caption{test management function test}
    \label{fig:test management function test}
\end{figure}
\newpage
\subsubsection{测试运行管理}
在本测试模块中,主要验证了测试运行管理机制的效率与准确性。通过定义多个安全和不安全的测试用例,分别模拟了不同执行时间、不同任务类型的测试场景。每个测试用例通过sleep函数模拟了不同的执行时间,以测试并行执行和顺序执行的性能差异。
测试结果解释如下
\begin{itemize}
    \item 使用一个具有八个线程的线程池,并行执行了三个测试用例
    \item 顺序执行了三个测试用例
\end{itemize}
\begin{framed}
    \begin{lstlisting}[language=C++]
ZTEST_F(RUN, safe_test_single_case1, safe) {
  sleep(2);
  ASSERT_TRUE(true);
  return ZState::z_success;
}
ZTEST_F(RUN, safe_test_single_case2, safe) {
  sleep(1);
  ASSERT_TRUE(true);
  return ZState::z_success;
}
ZTEST_F(RUN, safe_test_single_case3, safe) {
  sleep(3);
  ASSERT_TRUE(true);
  return ZState::z_success;
}
ZTEST_F(RUN, unsafe_test_single_case1, unsafe) {
  sleep(1);
  EXPECT_EQ(false, false);
  return ZState::z_success;
}
ZTEST_F(RUN, unsafe_test_single_case2, unsafe) {
  sleep(2);
  EXPECT_EQ(false, false);
  return ZState::z_success;
}
ZTEST_F(RUN, unsafe_test_single_case3, unsafe) {
  sleep(3);
  EXPECT_EQ(false, false);
  return ZState::z_success;
}
\end{lstlisting}
\end{framed}
\begin{figure}[H]
    \centering
    \includegraphics[width=\textwidth]{img/context.png}
    \caption{test 执行器 function test}
    \label{fig:test 执行器 function test}
\end{figure}

\subsubsection{数据驱动测试}
展示了两种不同类型的数据驱动测试用例,分别使用了内存中的数据集以及外部CSV文件作为测试数据源。

\begin{framed}
    \begin{lstlisting}[language=C++]
ZTestDataManager<vector<int>, int> sum_test_data = {
    {{1, 2}, 3}, {{-1, 1}, 0}, {{10, 20}, 30}};
ZTEST_P(ArithmeticSuite, SumTest, sum_test_data) {
  auto &&[inputs, expected] = _data.current();
  int actual = inputs[0] + inputs[1];
  EXPECT_EQ_FOREACH(expected, actual);
  return ZState::z_success;
}
ZTestDataManager<tuple<float, int>, float> sum_test_data2 = {
    {{1.2, 2}, 3.2}, {{-1.0, 1}, 0.0}, {{10.1, 20}, 30.2}};
ZTEST_P(ArithmeticSuite, SumTestfordiff, sum_test_data2) {
  auto &&[inputs, expected] = _data.current();
  float actual = std::get<0>(inputs) + std::get<1>(inputs);
  EXPECT_EQ_FOREACH(expected, actual);
  return ZState::z_success;
}
ZTEST_P_CSV(MathTests, AdditionTests, "data.csv") {
  auto inputs = getInput();
  auto expected = getOutput();
  double actual = std::get<double>(inputs[0]) + std::get<double>(inputs[1]);
  EXPECT_EQ(actual, std::get<double>(expected));
  return ZState::z_success;
}
\end{lstlisting}
\end{framed}
\begin{figure}[H]
    \centering
    \includegraphics[width=\textwidth]{img/data.png}
    \caption{数据驱动 function test}
    \label{fig: 数据驱动 function test}
\end{figure}
\subsubsection{benchmark测试}
测试benchmark类型测试的定义,以及测试时间分布的可视化,以及在GUI上对于CPU和内存使用率的监视。

\begin{framed}

    \begin{lstlisting}[language=C++]
ZBENCHMARK(Vector, PushBack) {
  std::vector<int> v;
  for (int i = 0; i < 10000; ++i)
    v.push_back(i);
  return ZState::z_success;
}
ZBENCHMARK(Matrix, PushBack, 20000) {
  std::vector<int> v;
  for (int i = 0; i < 1000; ++i)
    v.push_back(random());
  return ZState::z_success;
}\end{lstlisting}
\end{framed}
\begin{figure}[H]
    \centering
    \includegraphics[width=\textwidth]{img/ben.png}
    \caption{数据驱动 function test}
    \label{fig: 数据驱动 function test}
\end{figure}

\begin{itemize}
    \item
\end{itemize}
\subsubsection{报告生成测试}

测试了html,json,xml格式报告的生成。
\begin{figure}[H]
    \centering
    \includegraphics[width=0.8\textwidth]{img/html.png}
    \caption{HTML 测试报告}
    \label{fig: 数据驱动 function test}
\end{figure}
\begin{figure}[H]
    \centering
    \includegraphics[width=0.6\textwidth]{img/json.png}
    \caption{部分json报告}
    \label{fig: 数据驱动 function test}
\end{figure}

\begin{figure}[H]
    \centering
    \includegraphics[width=0.6\textwidth]{img/xml.png}
    \caption{xml(JUnit格式)报告}
    \label{fig: 数据驱动 function test}
\end{figure}
\subsubsection{GUI测试}
测试了GUI的展示,主题的切换,AI助手展示等功能。
\begin{figure}[H]
    \centering
    \begin{minipage}{0.5\textwidth}
        \includegraphics[width=\textwidth]{img/showgui.png}
        \caption{Dark theme GUI展示}
        \label{fig:showgui}
    \end{minipage}%
    \begin{minipage}{0.5\textwidth}
        \includegraphics[width=\textwidth]{img/light.png}
        \caption{Light Theme GUI展示}
        \label{fig:light}
    \end{minipage}
\end{figure}
\begin{figure}[H]
    \centering
    \includegraphics[width=0.4\textwidth]{img/aih.png}
    \caption{AI诊断助手展示}
    \label{fig: 数据驱动 function test}
\end{figure}
\subsubsection{AI诊断测试}
生成报告的时候,调用AI(qwen turbo)进行诊断,并写入到HTML报告中。
\begin{figure}[H]
    \centering
    \includegraphics[width=\textwidth]{img/ai.png}
\end{figure}
在GUI界面,调用AI(qwen turbo)进行诊断对于单个测试用例进行诊断并显示,允许用户输入自定义的提示词和测试所在的文件路径。
\begin{figure}[H]
    \centering
    \begin{minipage}{0.45\textwidth}
        \includegraphics[width=\textwidth]{img/aih1.png}
    \end{minipage}
    \hfill
    \begin{minipage}{0.45\textwidth}
        \includegraphics[width=\textwidth]{img/aih2.png}
    \end{minipage}
\end{figure}


\section{个人总结}
\begin{tabular}{|p{50pt}|p{350pt}|}
    \hline
    人名  & 心得                                                                                                                                                                                                                                                                                \\ \hline
    郑辰阳 & 测试框架设计实现过程深刻体会模块化设计和分层架构重要性,引入 MVC 模式实现模型、视图、控制器清晰分离,提高代码可维护性和扩展性。多线程环境下优化资源监控和任务调度,保障系统稳定性和性能,结合 AI 分析能力,为测试结果提供智能化反馈,增强工具实用性,开发过程强化技术深度,更注重团队协作与沟通效率,未来将持续迭代,完善功能,提升用户体验。                                                                                                       \\ \hline
    叶穗华 & 本次 program,主要负责前端部分,选择 imgui 库进行前端编写,顺利完成前端架构和绘制,体会到前端结构需依使用者需求设计,要考虑跨平台、库文件版本等因素。                                                                                                                                                                                                \\ \hline
    齐彦松 & 参与 “ztest” 单元测试框架开发,感受所学知识能用于游戏制作和工具软件设计, ztest 作为轻量、易用的 C++ 测试框架,解决了 Google Test 等现有框架在 GUI 支持、并发测试和报告系统上的不足,通过创新设计提升用户体验和扩展性。欣赏团队实现多种测试用例创建方式,体现 “用户友好” 理念,多线程安全评测设计使其意识到并发问题复杂性及分类处理重要性,通过实践深入了解 C++ 现代特性和设计模式,此次经历提升技术能力,认识到优秀软件产品需兼顾技术深度和用户体验,团队高效沟通协作是关键,期待继续探索工程实践挑战与解决方案。 \\ \hline
    王瑞箐 & 经历本次大作业成长颇多,这是一规模较大、复杂程度高的项目,面临很大挑战,因时间问题只能实践中学习,虽仓促、对很多东西一知半解,但也收获众多知识,掌握新技能,锻炼处理问题能力。                                                                                                                                                                                           \\ \hline
    吴泓庆 & 在 zTest 项目负责 GUI 设计工作,技术上掌握 MVC 架构实践运用,借助 ImGui 框架实现可视化界面,设计简单可视化窗口,协调 ZTestModel、ZTestView、ZTestController 3 个类处理数据可视化,提升用户友好度。团队协作中提高团队合作能力,与团队程序确定接口推进项目,规范项目不同类实现,防止编译错误,降低理解难度。                                                                                              \\ \hline
\end{tabular}

\section{参考文献}
\begin{thebibliography}{99}
    \bibitem{gtest} Google Test. \textit{Google Test Documentation}. [Online]. Available: https://github.com/google/googletest
    \bibitem{junit} JUnit. \textit{JUnit Documentation}. [Online]. Available: https://junit.org/junit5/
    \bibitem{catch} Catch2. \textit{Catch2 Documentation}. [Online]. Available: https://github.com/catchorg/Catch2
    \bibitem{gtest-parallel} Google Test Parallel. \textit{Google Test Parallel Documentation}. [Online]. Available: https://github.com/google/gtest-parallel
    \bibitem{pytest} Pytest. \textit{Pytest Documentation}. [Online]. Available: https://docs.pytest.org/en/latest/
    \bibitem{miniunit} MiniUnit. \textit{MiniUnit Documentation}. [Online]. Available: https://github.com/urin/miniunit
\end{thebibliography}

\end{document}